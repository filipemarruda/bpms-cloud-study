%% abtex2-modelo-trabalho-academico.tex, v-1.6 laurocesar
%% Copyright 2012-2013 by abnTeX2 group at http://abntex2.googlecode.com/
%%
%% This work may be distributed and/or modified under the
%% conditions of the LaTeX Project Public License, either version 1.3
%% of this license or (at your option) any later version.
%% The latest version of this license is in
%%   http://www.latex-project.org/lppl.txt
%% and version 1.3 or later is part of all distributions of LaTeX
%% version 2005/12/01 or later.
%%
%% This work has the LPPL maintenance status `maintained'.
%%
%% The Current Maintainer of this work is the abnTeX2 team, led
%% by Lauro César Araujo. Further information are available on
%% http://abntex2.googlecode.com/
%%
%% This work consists of the files abntex2-modelo-trabalho-academico.tex,
%% abntex2-modelo-include-comandos and abntex2-modelo-references.bib
%%

% ------------------------------------------------------------------------
% ------------------------------------------------------------------------
% abnTeX2: Modelo de Trabalho Academico (tese de doutorado, dissertacao de
% mestrado e trabalhos monograficos em geral) em conformidade com
% ABNT NBR 14724:2011: Informacao e documentacao - Trabalhos academicos -
% Apresentacao
% ------------------------------------------------------------------------
% ------------------------------------------------------------------------

% verso e anverso:
\documentclass[12pt,openright,twoside,a4paper,english]{abntex2}

% apenas verso:	
% \documentclass[12pt,oneside,a4paper,english,french,spanish]{abntex2}


% ---
% PACOTES
% ---

% ---
% Pacotes fundamentais
% ---
\usepackage{cmap}				% Mapear caracteres especiais no PDF
\usepackage{lmodern}			% Usa a fonte Latin Modern			
\usepackage[T1]{fontenc}		% Selecao de codigos de fonte.
\usepackage[utf8]{inputenc}		% Codificacao do documento (conversão automática dos acentos)
\usepackage{lastpage}			% Usado pela Ficha catalográfica
\usepackage{indentfirst}		% Indenta o primeiro parágrafo de cada seção.
\usepackage{color}				% Controle das cores
\usepackage{graphicx}			% Inclusão de gráficos
\usepackage{longtable}			% Tabelas longas
\usepackage{mdframed}
\usepackage{nomencl}
\usepackage{fumec} %
% ---
		
% ---
% Pacotes de citações
% ---
%\usepackage[brazilian,hyperpageref]{backref}	 % Paginas com as citações na bibl
\usepackage[alf]{abntex2cite}	% Citações padrão ABNT

% ---
% CONFIGURAÇÕES DE PACOTES
% ---

%% ---
%% Configurações do pacote backref
%% Usado sem a opção hyperpageref de backref
%\renewcommand{\backrefpagesname}{Citado na(s) página(s):~}
%% Texto padrão antes do número das páginas
%\renewcommand{\backref}{}
%% Define os textos da citação
%\renewcommand*{\backrefalt}[4]{
%	\ifcase #1 %
%		Nenhuma citação no texto.%
%	\or
%		Citado na página #2.%
%	\else
%		Citado #1 vezes nas páginas #2.%
%	\fi}%
%% ---

% ----------------
% Novos comandos
% ----------------

\makenomenclature

% desenhar tabela escala likert 5 pontos
\newcommand{\tabelaLikert} [5] {
	\vspace{-0.5cm}
	\begin{table*}[h]
		\centering
		\begin{tabular}{|l|l|l|l|l|}
			\hline
			#1 & #2 & #3 & #4 & #5 \\
			\hline
				&	&	&	& 	\\
			\hline
		\end{tabular}
	\end{table*}
}


\newcommand{\tabelaSimNao} [4] {
	\vspace{-0.5cm}
	\begin{table*}[h]
		\centering
		\begin{tabular}{|l|r|}
			\hline
			#1 & #2 \\
			\hline
			#3 & #4 \\
			\hline
		\end{tabular}
	\end{table*}
}
\newcommand{\caixaTexto}{
	\vspace{-1cm}
	\begin{center}
		\fbox{
			\begin{minipage}{0.95\textwidth}
				\hfill \vspace{0.5cm}
			\end{minipage}
		}
	\end{center}
}

% ---
% Informações de dados para CAPA e FOLHA DE ROSTO
% ---
\titulo{Sistema de Gerenciamento de Processos de Negócio(BPMS) implementado usando cloud computing.}
\autor{Filipe Mendes Arruda}
\local{Belo Horizonte}
\data{2014}
\orientador{Prof. Rafael Nunes Linhares Papa}
%\coorientador{}
\tipotrabalho{Trabalho de Conclusão de Curso}
% O preambulo deve conter o tipo do trabalho, o objetivo,
% o nome da instituição e a área de concentração
\preambulo{Trabalho de Conclusão de Curso apresentado ao curso de Ciência da Computação como requisito parcial para obtenção do título de Bacharel em Ciência da Computação.}
% ---

% ---
% Configurações de aparência do PDF final

% alterando o aspecto da cor azul
\definecolor{blue}{RGB}{41,5,195}

% informações do PDF
\makeatletter
\hypersetup{
     	%pagebackref=true,
		pdftitle={\@title},
		pdfauthor={\@author},
    	pdfsubject={\imprimirpreambulo},
	    pdfcreator={LaTeX with abnTeX2},
		pdfkeywords={abnt}{latex}{abntex}{abntex2}{trabalho acadêmico},
		colorlinks=true,       		% false: boxed links; true: colored links
    	linkcolor=blue,          	% color of internal links
    	citecolor=blue,        		% color of links to bibliography
    	filecolor=magenta,      		% color of file links
		urlcolor=blue,
		bookmarksdepth=4
}
\makeatother
% ---

% ---
% Espaçamentos entre linhas e parágrafos
% ---

% O tamanho do parágrafo é dado por:
\setlength{\parindent}{1.3cm}

% Controle do espaçamento entre um parágrafo e outro:
\setlength{\parskip}{0.2cm}  % tente também \onelineskip

% ---
% compila o indice
% ---
\makeindex
% ---

% ----
% Início do documento
% ----
\begin{document}

% Retira espaço extra obsoleto entre as frases.
\frenchspacing

% ----------------------------------------------------------
% ELEMENTOS PRÉ-TEXTUAIS
% ----------------------------------------------------------
\pretextual

% ---
% Capa
% ---
\imprimircapa
% ---

% ---
% Folha de rosto
% (o * indica que haverá a ficha bibliográfica)
% ---
\imprimirfolhaderosto


% resumo em português
\begin{resumo}[Resumo]
% Segundo a \citeonline[3.1-3.2]{NBR6028:2003}, o resumo deve ressaltar o  objetivo, o método, os resultados e as conclusões do documento. A ordem e a extensão  destes itens dependem do tipo de resumo (informativo ou indicativo) e do  tratamento que cada item recebe no documento original. O resumo deve ser  precedido da referência do documento, com exceção do resumo inserido no  próprio documento. (\ldots) As palavras-chave devem figurar logo abaixo do  resumo, antecedidas da expressão Palavras-chave:, separadas entre si por  ponto e finalizadas também por ponto.

 \vspace{\onelineskip}
 \noindent
 \textbf{Palavras-chaves}: Gerenciamento de Processos de Negócio; BPMS; Computação em núvem; SWF; Amazon Simple Workflow Framework.
\end{resumo}
% ---

% ---
% inserir lista de ilustrações
% ---
\pdfbookmark[0]{\listfigurename}{lof}
\listoffigures*
\cleardoublepage
% ---

% ---
% inserir lista de tabelas
% ---
%\pdfbookmark[0]{\listtablename}{lot}
%\listoftables*
%\cleardoublepage
% ---

% ---
% inserir lista de abreviaturas e siglas
% ---
%\renewcommand{\nomname}{\listadesiglasname} \pdfbookmark[0]{\nomname}{las}
%\printnomenclature
%\cleardoublepage
% ---

% ---
% inserir lista de símbolos
% ---
%\begin{simbolos}
%  \item[$ \Gamma $] Letra grega Gama
%  \item[$ \Lambda $] Lambda
%  \item[$ \zeta $] Letra grega minúscula zeta
%  \item[$ \in $] Pertence
%\end{simbolos}
% ---

% ---
% inserir o sumario
% ---
\pdfbookmark[0]{\contentsname}{toc}
\tableofcontents*
\cleardoublepage
% ---

% ----------------------------------------------------------
% ELEMENTOS TEXTUAIS
% ----------------------------------------------------------
\textual

% ----------------------------------------------------------
% Introdução
% ----------------------------------------------------------
%\chapter*[Introdução]{Introdução}
%\addcontentsline{toc}{chapter}{Introdução}  % isso faz com que apareça no sumário
\chapter[introducao]{Introdução}

A engenharia de sistemas embora seja uma ciência "relativamente" recente, aproximadamente 1940 \cite{EngSisUFMG}, representa um grande marco para toda uma era. Através dela e de sua base interdisciplinar torna-se possível o desenvolvimento de projetos com níveis de complexidade muito maiores que sem ela provavelmente não seriam alcançáveis.

\begin{quote}

O termo "Engenharia de Sistemas" parece ter se originado na década dos 1940, dentro dos Bell Telephone Laboratories, já dotado do sentido que é atualmente consagrado, designando a área de conhecimento que lida com os aspectos de sistematização e validação do projeto de sistemas tecnológicos de elevada complexidade (no sentido de agregarem um elevado número de sub-sistemas de diferentes níveis lógicos).

A Engenharia de Sistemas integra diferentes disciplinas e especialidades em uma equipe de projeto, formando um processo de desenvolvimento estruturado que se estende do conceito ao projeto, e deste à operação. A Engenharia de Sistemas considera tanto as questões de ordem econômica quanto técnica, com o objetivo de gerar produtos de qualidade que atendam às necessidades dos consumidores.

\cite{EngSisUFMG}

\end{quote}

Analisando o contexto atual da engenharia de sistemas, surge um novo tema que intriga pela proposta audaciosa que tem. A Engenharia de Sistemas Conduzida por Modelos(\nomenclature{MBSE}{Model-Based Systems Engineering}) é a aplicação formalizada de modelagem para auxílio na definição de requisitos, projeto, análise, verificação e validação desde a fase de concepção, construção até o fim do ciclo de desenvolvimento do sistema.\cite{INCOSE200402}

Embora existam muitos trabalhos já publicados sobre a MBSE, uma visão consistente do impacto dela ainda falta e é esta a motivação deste estudo.

O objetivo deste estudo é fazer uma análise da literatura existente de MBSE elegendo alguns artigos e buscar nestes alguns padrões de aplicação da Modelagem de Sistemas Conduzida por Modelos.
% ----------------------------------------------------------
% Referencial Teórico
% ----------------------------------------------------------
\chapter{Referencial Teórico}

\section{Gerenciamento de Processos}

\section{BPMN}

\section{BPMS}

\section{Cloud Computing}

%% ---- https://aws.amazon.com/what-is-cloud-computing/?nc1=h_l2_cc
Computação em nuvem, por definição, se refere a entrega sobre demanda de recursos computacionais e aplicações via internet pagando de acordo com o uso.

%% ---- https://aws.amazon.com/what-is-cloud-computing/?nc1=h_l2_cc
A nuvem computacional fornece rápido acesso a preços baixos e flexíveis a recursos computacionais. Utilizando \textit{cloud computing} não é necessário fazer grandes investimentos iniciais em hardware e gastar uma grande quantidade de tempo ajustando e gerenciando estes equipamentos. Ao invés disso pode-se contratar o tipo e tamanho certo dos recursos necessários para o projeto em questão.


\section{Amazon Web Services}

\section{Amazon Simple Workflow Framework}

\subsection{Estrutura da aplicação}

Conceitualmente, uma aplicação usando AWS Flow Framework consiste em três componentes básicos, um \textit{workflow starter}, um \textit{workflow worker} e um \textit{activities worker}.

Este diagrama representa uma aplicação AWS Flow Framework básica:

\begin{figure}[htp]
	\centering
	\includegraphics[width=10cm]{imagens/application-model.png}
	\caption{Estrutura da aplicação}
	\label{fig:application-model}
\end{figure}

\subsection{Workflow Starter}

\subsection{Workflow Worker}

\subsection{Activities Worker}

\section{Estudo de caso}

\section{Protótipo}

% ----------------------------------------------------------
% Metodologia
% ----------------------------------------------------------
\chapter{Metodologia}

Podemos definir pesquisa como um procedimento que tem como objetivo proporcionar respostas a um ou vários problemas propostos. Uma pesquisa é cabível quando não se dispõe de informação suficiente para responder ao problema ou quando as informações que teoricamente poderiam chegar a tal resposta encontram-se em um estado de desordem que não possam ser adequadamente relacionadas àquele.

A pesquisa é desenvolvida usando-se métodos, técnicas e outros procedimentos científicos. Ela desenvolve-se ao longo de um processo que envolve inúmeras fases, exigindo-se classificação para que possamos enxergar de forma racional tais partes do processo. Classificaremos nossa pesquisa segundo objetivos, fontes, procedimentos técnicos e forma de abordagem.

\section{Segundo os objetivos}

Com base no que diz \cite{AntonioCarlosGil}, segundo os objetivos faremos neste uma pesquisa de caráter descritivo caracterizado pela revisão sistemática de artigos anteriormente publicados sobre o tema em questão.

"As pesquisas descritivas têm como objetivo primordial a descrição das características de determinada população ou fenômeno ou, então, o estabelecimento de relações entre variáveis."\cite[p.44]{AntonioCarlosGil}

\section{Segundo as fontes}

Considerando que planejamos fazer uma revisão de trabalhos anteriormente realizados sobre o tema em questão, este documento visa realizar uma pequisa bibliográfica com relação às fontes.

"A pesquisa bibliográfica é desenvolvida com base em material já elaborado, constituído principalmente de livros e artigos científicos."\cite{AntonioCarlosGil}


\section{Segundo os procedimentos}

Segundo os procedimentos, faremos uma pesquisa bibliográfica e um estudo de caso.

A pesquisa bibliográfica é, segundo \cite{AntonioCarlosGil}, desenvolvida com base em  material já elaborado, constituído principalmente de livros e artigos científicos.

\section{Segundo a forma de abordagem}

Segundo a forma de abordagem faremos uma pesquisa qualitativa que não requer o uso de métodos e técnicas estatísticas. O material existente é a fonte direta para coleta de dados sendo 
descritiva tendemos a analisar os dados indutivamente. O processo e seu significado são os focos principais de abordagem. \cite{TatianaEng}



% ----------------------------------------------------------
% Conclusão
% ----------------------------------------------------------
%\chapter*[Introdução]{Introdução}
%\addcontentsline{toc}{chapter}{Introdução}  % isso faz com que apareça no sumário
\chapter{Conclusão}

% ----------------------------------------------------------
% ELEMENTOS PÓS-TEXTUAIS
% ----------------------------------------------------------
\postextual


% ----------------------------------------------------------
% Referências bibliográficas
% ----------------------------------------------------------
\bibliography{referencias}

% ----------------------------------------------------------
% Glossário
% ----------------------------------------------------------
%
% Consulte o manual da classe abntex2 para orientações sobre o glossário.
%
%\glossary

% ----------------------------------------------------------
% Apêndices
% ----------------------------------------------------------

%    % ---
%    % Inicia os apêndices
%    % ---
%    \begin{apendicesenv}
%
%    % Imprime uma página indicando o início dos apêndices
%    \partapendices
%
%    \end{apendicesenv}
%    % ---
%
%
% ----------------------------------------------------------
% Anexos
% ----------------------------------------------------------

%    % ---
%    % Inicia os anexos
%    % ---
%    \begin{anexosenv}
%    %
%    %% Imprime uma página indicando o início dos anexos
%    \partanexos
%    %
%    \end{anexosenv}
%
%---------------------------------------------------------------------
% INDICE REMISSIVO
%---------------------------------------------------------------------

\printindex

\end{document}
