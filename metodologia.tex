% ----------------------------------------------------------
% Metodologia
% ----------------------------------------------------------
\chapter{Metodologia}

Podemos definir pesquisa como um procedimento que tem como objetivo proporcionar respostas a um ou vários problemas propostos. Uma pesquisa é cabível quando não se dispõe de informação suficiente para responder ao problema ou quando as informações que teoricamente poderiam chegar a tal resposta encontram-se em um estado de desordem que não possam ser adequadamente relacionadas àquele.

A pesquisa é desenvolvida usando-se métodos, técnicas e outros procedimentos científicos. Ela desenvolve-se ao longo de um processo que envolve inúmeras fases, exigindo-se classificação para que possamos enxergar de forma racional tais partes do processo. Classificaremos nossa pesquisa segundo objetivos, fontes, procedimentos técnicos e forma de abordagem.

\section{Segundo os objetivos}

Com base no que diz \cite{AntonioCarlosGil}, segundo os objetivos faremos neste uma pesquisa de caráter descritivo caracterizado pela revisão sistemática de artigos anteriormente publicados sobre o tema em questão.

As pesquisas descritivas têm como objetivo primordial a descrição das características de determinada população ou fenômeno ou, então, o estabelecimento de relações entre variáveis.\cite[p.44]{AntonioCarlosGil}

\section{Segundo as fontes}

Considerando que planejamos fazer uma revisão de trabalhos anteriormente realizados sobre o tema em questão, este documento visa realizar uma pequisa bibliográfica com relação às fontes.

A pesquisa bibliográfica é desenvolvida com base em material já elaborado, constituído principalmente de livros e artigos científicos.\cite{AntonioCarlosGil}


\section{Segundo os procedimentos}

Segundo os procedimentos, faremos uma pesquisa bibliográfica que é, segundo \cite{AntonioCarlosGil}, desenvolvida com base em  material já elaborado, constituído principalmente de livros e artigos científicos.

\section{Segundo a forma de abordagem}

Segundo a forma de abordagem faremos uma pesquisa qualitativa que não requer o uso de métodos e técnicas estatísticas. O material existente é a fonte direta para coleta de dados sendo 
descritiva tendemos a analisar os dados indutivamente. O processo e seu significado são os focos principais de abordagem. \cite{TatianaEng}

