% ----------------------------------------------------------
% Referencial Teórico
% ----------------------------------------------------------
\chapter{Referencial Teórico}

\section{Gerenciamento de Processos}

\section{BPMN}

\section{BPMS}

\section{Cloud Computing}

%% ---- https://aws.amazon.com/what-is-cloud-computing/?nc1=h_l2_cc
Computação em nuvem, por definição, se refere a entrega sobre demanda de recursos computacionais e aplicações via internet pagando de acordo com o uso.

%% ---- https://aws.amazon.com/what-is-cloud-computing/?nc1=h_l2_cc
A nuvem computacional fornece rápido acesso a preços baixos e flexíveis a recursos computacionais. Utilizando \textit{cloud computing} não é necessário fazer grandes investimentos iniciais em hardware e gastar uma grande quantidade de tempo ajustando e gerenciando estes equipamentos. Ao invés disso pode-se contratar o tipo e tamanho certo dos recursos necessários para o projeto em questão.



\section{Amazon Web Services}

\section{Amazon Simple Workflow Framework}

\section{Estudo de caso}

\section{Protótipo}

De acordo com o \cite{INCOSE200402}, a Engenharia de Sistemas é uma abordagem interdisciplinar que torna possível a concretização de "Sistemas" com elevada complexidade. O seu foco encontra-se em definir, de maneira precoce no ciclo de desenvolvimento de um sistema, as necessidades do usuário, bem como as funcionalidades requeridas, realizando a documentação sistemática dos requisitos, e abordando a síntese de projeto e a etapa de validação considerando o problema por completo: operação; custos e cronogramas; performance; treinamento e suporte; teste; instalação; fabricação;

A forma atual de diversos métodos que hoje integram a Engenharia de Sistemas teve um marco importante que faz parte dos dias atuais de praticamente todas as grandes cidades do mundo, a concepção do Boeing 777, concluído em 1995. O projeto seria desenvolvido para atender a demanda das empresas de aviação - o que significou integrar na equipe times de engenheiros das companhias clientes. Levou menos de cinco anos entre a especificação do produto e o primeiro voo, prazo inédito até então. \cite{EngSisUFMG}


Um modelo é uma aproximação, representação, ou idealização de um aspecto selecionado da estrutura, comportamento, operação, ou outra característica de um processo, conceito ou sistema do mundo real, por exemplo uma abstração.\cite{IEEE61012}

Um modelo usualmente oferece várias visões a fim de servir para diferentes propósitos. Uma visão é a representação de um sistema a partir da perspectiva das questões ou preocupações em questão. \cite{IEEE1471}

A Engenharia de Sistemas Conduzida por Modelos(\nomenclature{MBSE}{Model-Based Systems Engineering}) é a aplicação formalizada de modelagem para auxílio do início ao fim do ciclo de desenvolvimento de um sistema.\cite{INCOSE200402}


Com base nos estudos eleitos selecionaremos algumas das aplicações da MBSE encontradas para que possamos analisá-las e nesta seção descrevê-las com detalhes suficientes.


Analisando os estudos citados na seção anterior faremos um levantamento de alguns padrões observados na aplicação da MBSE na indústria em geral.