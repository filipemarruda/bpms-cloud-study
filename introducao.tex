% ----------------------------------------------------------
% Introdução
% ----------------------------------------------------------
%\chapter*[Introdução]{Introdução}
%\addcontentsline{toc}{chapter}{Introdução}  % isso faz com que apareça no sumário
\chapter[introducao]{Introdução}

A engenharia de sistemas embora seja uma ciência "relativamente" recente, aproximadamente 1940 \cite{EngSisUFMG}, representa um grande marco para toda uma era. Através dela e de sua base interdisciplinar torna-se possível o desenvolvimento de projetos com níveis de complexidade muito maiores que sem ela provavelmente não seriam alcançáveis.

\begin{quote}

O termo "Engenharia de Sistemas" parece ter se originado na década dos 1940, dentro dos Bell Telephone Laboratories, já dotado do sentido que é atualmente consagrado, designando a área de conhecimento que lida com os aspectos de sistematização e validação do projeto de sistemas tecnológicos de elevada complexidade (no sentido de agregarem um elevado número de sub-sistemas de diferentes níveis lógicos).

A Engenharia de Sistemas integra diferentes disciplinas e especialidades em uma equipe de projeto, formando um processo de desenvolvimento estruturado que se estende do conceito ao projeto, e deste à operação. A Engenharia de Sistemas considera tanto as questões de ordem econômica quanto técnica, com o objetivo de gerar produtos de qualidade que atendam às necessidades dos consumidores.

\cite{EngSisUFMG}

\end{quote}

Analisando o contexto atual da engenharia de sistemas, surge um novo tema que intriga pela proposta audaciosa que tem. A Engenharia de Sistemas Conduzida por Modelos(\nomenclature{MBSE}{Model-Based Systems Engineering}) é a aplicação formalizada de modelagem para auxílio na definição de requisitos, projeto, análise, verificação e validação desde a fase de concepção, construção até o fim do ciclo de desenvolvimento do sistema.\cite{INCOSE200402}

Embora existam muitos trabalhos já publicados sobre a MBSE, uma visão consistente do impacto dela ainda falta e é esta a motivação deste estudo.

O objetivo deste estudo é fazer uma análise da literatura existente de MBSE elegendo alguns artigos e buscar nestes alguns padrões de aplicação da Modelagem de Sistemas Conduzida por Modelos.